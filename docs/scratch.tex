% Created 2023-08-21 Mon 09:55
% Intended LaTeX compiler: pdflatex
\documentclass[11pt]{article}
\usepackage[utf8]{inputenc}
\usepackage[T1]{fontenc}
\usepackage{graphicx}
\usepackage{longtable}
\usepackage{wrapfig}
\usepackage{rotating}
\usepackage[normalem]{ulem}
\usepackage{amsmath}
\usepackage{amssymb}
\usepackage{capt-of}
\usepackage{hyperref}
\author{Biggie Dickus}
\date{\today}
\title{}
\hypersetup{
 pdfauthor={Biggie Dickus},
 pdftitle={},
 pdfkeywords={},
 pdfsubject={},
 pdfcreator={Emacs 28.2 (Org mode 9.5.5)}, 
 pdflang={English}}
\begin{document}

\tableofcontents

\section{Tokens}
\label{sec:orgc250f3c}
\subsection{Breking tokens}
\label{sec:org1186f84}
pensiamo innazitutto a come si potrebbe definire un token, potremmo definirlo come
\subsubsection{Massimale?}
\label{sec:org386b0f1}
\begin{quote}
famo che un token come una sequenza massimale di caratteri o tutti speciali o tutti normali
\end{quote}
in questo caso una possibile implementazione sarebbe
\begin{verbatim}
tokens = foldr (++) [] -- take the twice nested and "flatten" into a once nested
  . map (groupBy ((==) `on` isSpecial)) -- separate special characters
  . words -- break on whitespace
isSpecial c = elem c "(),`'"
\end{verbatim}

questa purtroppo non non funziona in casi quali, ad esempioo
\begin{verbatim}
tokens "(defun il-cazzo (che me frega) `(walu ,igi)))"
\end{verbatim}

si nota infatti, che il risultato di questa viene
\begin{verbatim}
["(","defun","il-cazzo","(","che","me","frega",")","`(","walu",",","igi",")))"]
\end{verbatim}

si noti soprattutto il \texttt{"`("} , questo non vale come token, e vorremmo separarlo nei due token \texttt{"`"}, e \texttt{"("}

\subsubsection{Logica di Break}
\label{sec:orgf85d3a6}
possiamo definire una qualche logica di break per cui, andando da sinistra a destra col classicone (un mangialiste)
\begin{itemize}
\item se stiamo "costruendo" un token speciale (se \(\exists\ spe \in speciali\ t.c.\ spe \text{ prefisso di } sta\ cazzo\ di\ stringa\)) allora
\begin{itemize}
\item se il prossimo carattere continua la cosa, continua
\item altrimenti spacca
\end{itemize}
\item se non stamo costruendo un token speciale allora
\begin{itemize}
\item se incontri un carattere speciale allora spaccca
\item se incontri uno spazio allora spacca
\end{itemize}
\item se siamo su uno spazio bianco
\begin{itemize}
\item snobbalo
\end{itemize}
\end{itemize}

devo dire non mi sarei mai aspettato di scrivere pseudocodice hakell con delle liste,
devo rivedere lisp sti giorni, altrimenti cazzo se divento un riso al curry\footnote{noto anche come xmonad}

\begin{enumerate}
\item Un minimo di astrazione
\label{sec:orgaa0480a}
ai fini di rompermi meno il cazzo con certe cose, visto che tanto la logica di break e la logica di ignoramus fanno il tokenizer, facciamo prima a specificare le due logiche, e fare un \texttt{definisciTokenizer <quella di break> <quella di ignorare>}
così non dobbiamo romperci il cazzo a riscrivere tutto il tokenizer ogni volta che ne cambia una.

ai fini di fare sta cosa, eccovi una funzione \emph{di ordine superiore} idiota
\begin{verbatim}
-- brk :  break, separa il primo token dal resto della stringa
-- ignr : ignore, ignora eventuali caratteri inutili all'inizio del resto
-- parecchio ad hoc
collectBreakIgnore :: ([a] -> ([a],[a])) -> ([a] -> [a]) -> [a] -> [[a]]
collectBreakIgnore _ _ [] = []
collectBreakIgnore brk ignr lst = let (a,b) = (brk lst)
                                  in a : (collectBreakIgnore brk ignr (ignr b))
\end{verbatim}

si descrive abbastanza da sola
ecco un'implementazione di \texttt{words} usando questa cosa, per dare un'idea

\begin{verbatim}
words' = collectBreakIgnore (break isSpace) ignoreWhite
  . ignoreWhite -- aggiunto per gestire leading whitspace
  where ignoreWhite :: String -> String 
        ignoreWhite = dropWhile isSpace
\end{verbatim}

\item Applicata alla logica di break
\label{sec:orgc7c3fcb}
qui ho fatto troppo il \texttt{REPL}-one per documentarla bene, rtfs
\end{enumerate}

\section{Tree}
\label{sec:orge537b53}
\texttt{Tree Be Determined}
\end{document}